\documentclass[]{article}
\usepackage{lmodern}
\usepackage{amssymb,amsmath}
\usepackage{ifxetex,ifluatex}
\usepackage{fixltx2e} % provides \textsubscript
\ifnum 0\ifxetex 1\fi\ifluatex 1\fi=0 % if pdftex
  \usepackage[T1]{fontenc}
  \usepackage[utf8]{inputenc}
\else % if luatex or xelatex
  \ifxetex
    \usepackage{mathspec}
  \else
    \usepackage{fontspec}
  \fi
  \defaultfontfeatures{Ligatures=TeX,Scale=MatchLowercase}
\fi
% use upquote if available, for straight quotes in verbatim environments
\IfFileExists{upquote.sty}{\usepackage{upquote}}{}
% use microtype if available
\IfFileExists{microtype.sty}{%
\usepackage{microtype}
\UseMicrotypeSet[protrusion]{basicmath} % disable protrusion for tt fonts
}{}
\usepackage[margin=1in]{geometry}
\usepackage{hyperref}
\hypersetup{unicode=true,
            pdftitle={Handson analysis on the POA accidents dataset},
            pdfauthor={Joao Pedro Oliveira},
            pdfborder={0 0 0},
            breaklinks=true}
\urlstyle{same}  % don't use monospace font for urls
\usepackage{color}
\usepackage{fancyvrb}
\newcommand{\VerbBar}{|}
\newcommand{\VERB}{\Verb[commandchars=\\\{\}]}
\DefineVerbatimEnvironment{Highlighting}{Verbatim}{commandchars=\\\{\}}
% Add ',fontsize=\small' for more characters per line
\usepackage{framed}
\definecolor{shadecolor}{RGB}{248,248,248}
\newenvironment{Shaded}{\begin{snugshade}}{\end{snugshade}}
\newcommand{\KeywordTok}[1]{\textcolor[rgb]{0.13,0.29,0.53}{\textbf{#1}}}
\newcommand{\DataTypeTok}[1]{\textcolor[rgb]{0.13,0.29,0.53}{#1}}
\newcommand{\DecValTok}[1]{\textcolor[rgb]{0.00,0.00,0.81}{#1}}
\newcommand{\BaseNTok}[1]{\textcolor[rgb]{0.00,0.00,0.81}{#1}}
\newcommand{\FloatTok}[1]{\textcolor[rgb]{0.00,0.00,0.81}{#1}}
\newcommand{\ConstantTok}[1]{\textcolor[rgb]{0.00,0.00,0.00}{#1}}
\newcommand{\CharTok}[1]{\textcolor[rgb]{0.31,0.60,0.02}{#1}}
\newcommand{\SpecialCharTok}[1]{\textcolor[rgb]{0.00,0.00,0.00}{#1}}
\newcommand{\StringTok}[1]{\textcolor[rgb]{0.31,0.60,0.02}{#1}}
\newcommand{\VerbatimStringTok}[1]{\textcolor[rgb]{0.31,0.60,0.02}{#1}}
\newcommand{\SpecialStringTok}[1]{\textcolor[rgb]{0.31,0.60,0.02}{#1}}
\newcommand{\ImportTok}[1]{#1}
\newcommand{\CommentTok}[1]{\textcolor[rgb]{0.56,0.35,0.01}{\textit{#1}}}
\newcommand{\DocumentationTok}[1]{\textcolor[rgb]{0.56,0.35,0.01}{\textbf{\textit{#1}}}}
\newcommand{\AnnotationTok}[1]{\textcolor[rgb]{0.56,0.35,0.01}{\textbf{\textit{#1}}}}
\newcommand{\CommentVarTok}[1]{\textcolor[rgb]{0.56,0.35,0.01}{\textbf{\textit{#1}}}}
\newcommand{\OtherTok}[1]{\textcolor[rgb]{0.56,0.35,0.01}{#1}}
\newcommand{\FunctionTok}[1]{\textcolor[rgb]{0.00,0.00,0.00}{#1}}
\newcommand{\VariableTok}[1]{\textcolor[rgb]{0.00,0.00,0.00}{#1}}
\newcommand{\ControlFlowTok}[1]{\textcolor[rgb]{0.13,0.29,0.53}{\textbf{#1}}}
\newcommand{\OperatorTok}[1]{\textcolor[rgb]{0.81,0.36,0.00}{\textbf{#1}}}
\newcommand{\BuiltInTok}[1]{#1}
\newcommand{\ExtensionTok}[1]{#1}
\newcommand{\PreprocessorTok}[1]{\textcolor[rgb]{0.56,0.35,0.01}{\textit{#1}}}
\newcommand{\AttributeTok}[1]{\textcolor[rgb]{0.77,0.63,0.00}{#1}}
\newcommand{\RegionMarkerTok}[1]{#1}
\newcommand{\InformationTok}[1]{\textcolor[rgb]{0.56,0.35,0.01}{\textbf{\textit{#1}}}}
\newcommand{\WarningTok}[1]{\textcolor[rgb]{0.56,0.35,0.01}{\textbf{\textit{#1}}}}
\newcommand{\AlertTok}[1]{\textcolor[rgb]{0.94,0.16,0.16}{#1}}
\newcommand{\ErrorTok}[1]{\textcolor[rgb]{0.64,0.00,0.00}{\textbf{#1}}}
\newcommand{\NormalTok}[1]{#1}
\usepackage{graphicx,grffile}
\makeatletter
\def\maxwidth{\ifdim\Gin@nat@width>\linewidth\linewidth\else\Gin@nat@width\fi}
\def\maxheight{\ifdim\Gin@nat@height>\textheight\textheight\else\Gin@nat@height\fi}
\makeatother
% Scale images if necessary, so that they will not overflow the page
% margins by default, and it is still possible to overwrite the defaults
% using explicit options in \includegraphics[width, height, ...]{}
\setkeys{Gin}{width=\maxwidth,height=\maxheight,keepaspectratio}
\IfFileExists{parskip.sty}{%
\usepackage{parskip}
}{% else
\setlength{\parindent}{0pt}
\setlength{\parskip}{6pt plus 2pt minus 1pt}
}
\setlength{\emergencystretch}{3em}  % prevent overfull lines
\providecommand{\tightlist}{%
  \setlength{\itemsep}{0pt}\setlength{\parskip}{0pt}}
\setcounter{secnumdepth}{0}
% Redefines (sub)paragraphs to behave more like sections
\ifx\paragraph\undefined\else
\let\oldparagraph\paragraph
\renewcommand{\paragraph}[1]{\oldparagraph{#1}\mbox{}}
\fi
\ifx\subparagraph\undefined\else
\let\oldsubparagraph\subparagraph
\renewcommand{\subparagraph}[1]{\oldsubparagraph{#1}\mbox{}}
\fi

%%% Use protect on footnotes to avoid problems with footnotes in titles
\let\rmarkdownfootnote\footnote%
\def\footnote{\protect\rmarkdownfootnote}

%%% Change title format to be more compact
\usepackage{titling}

% Create subtitle command for use in maketitle
\newcommand{\subtitle}[1]{
  \posttitle{
    \begin{center}\large#1\end{center}
    }
}

\setlength{\droptitle}{-2em}

  \title{Handson analysis on the POA accidents dataset}
    \pretitle{\vspace{\droptitle}\centering\huge}
  \posttitle{\par}
    \author{Joao Pedro Oliveira}
    \preauthor{\centering\large\emph}
  \postauthor{\par}
      \predate{\centering\large\emph}
  \postdate{\par}
    \date{October 31 2018}


\begin{document}
\maketitle

\subsubsection{This is my hands on analysis of the POA accidents
dataset.}\label{this-is-my-hands-on-analysis-of-the-poa-accidents-dataset.}

\paragraph{First, download the
dataset}\label{first-download-the-dataset}

\begin{Shaded}
\begin{Highlighting}[]
\NormalTok{file =}\StringTok{ "acidentes-2016.csv"}
\ControlFlowTok{if}\NormalTok{(}\OperatorTok{!}\KeywordTok{file.exists}\NormalTok{(file))\{}
  \KeywordTok{download.file}\NormalTok{(}\StringTok{"http://datapoa.com.br/storage/f/2017-08-03T13%3A19%3A45.538Z/acidentes-2016.csv"}\NormalTok{, }\DataTypeTok{destfile=}\NormalTok{file)}
\NormalTok{\}}
\end{Highlighting}
\end{Shaded}

\paragraph{Now, read the CSV file to a Dataframe using
readr}\label{now-read-the-csv-file-to-a-dataframe-using-readr}

\begin{Shaded}
\begin{Highlighting}[]
\KeywordTok{library}\NormalTok{(readr)}
\KeywordTok{library}\NormalTok{(RColorBrewer)}
\NormalTok{ac_data <-}\StringTok{ }\KeywordTok{read_delim}\NormalTok{(file, }\StringTok{";"}\NormalTok{)}
\end{Highlighting}
\end{Shaded}

\begin{verbatim}
## Parsed with column specification:
## cols(
##   .default = col_integer(),
##   LONGITUDE = col_double(),
##   LATITUDE = col_double(),
##   LOG1 = col_character(),
##   LOG2 = col_character(),
##   LOCAL = col_character(),
##   TIPO_ACID = col_character(),
##   LOCAL_VIA = col_character(),
##   DATA = col_date(format = ""),
##   DATA_HORA = col_datetime(format = ""),
##   DIA_SEM = col_character(),
##   HORA = col_time(format = ""),
##   TEMPO = col_character(),
##   NOITE_DIA = col_character(),
##   FONTE = col_character(),
##   BOLETIM = col_character(),
##   REGIAO = col_character(),
##   CONSORCIO = col_character()
## )
\end{verbatim}

\begin{verbatim}
## See spec(...) for full column specifications.
\end{verbatim}

\begin{Shaded}
\begin{Highlighting}[]
\NormalTok{ac_data}
\end{Highlighting}
\end{Shaded}

\begin{verbatim}
## # A tibble: 12,515 x 44
##        ID LONGITUDE LATITUDE LOG1  LOG2  PREDIAL1 LOCAL TIPO_ACID LOCAL_VIA
##     <int>     <dbl>    <dbl> <chr> <chr>    <int> <chr> <chr>     <chr>    
##  1 623243     -51.2    -30.1 R AR~ R CO~        0 Cruz~ ATROPELA~ R ARAPEI~
##  2 622413     -51.2    -30.1 R PA~ R JO~        0 Cruz~ ABALROAM~ R PADRE ~
##  3 622460     -51.2    -30.0 AV D~ <NA>         0 Logr~ ATROPELA~ AV DO LA~
##  4 622540     -51.2    -30.0 AV D~ R CA~        0 Cruz~ CHOQUE    AV DR NI~
##  5 622181     -51.1    -30.1 ESTR~ <NA>      8487 Logr~ CHOQUE    8487 EST~
##  6 622232     -51.2    -30.0 AV I~ <NA>       320 Logr~ COLISAO   320 AV I~
##  7 622414     -51.2    -30.1 R JO~ <NA>       965 Logr~ COLISAO   965 R JO~
##  8 622186     -51.2    -30.0 AV E~ <NA>       240 Logr~ ABALROAM~ 240 AV E~
##  9 622235     -51.2    -30.0 R GE~ <NA>      1445 Logr~ COLISAO   1445 R G~
## 10 622185     -51.2    -30.0 AV E~ <NA>         0 Logr~ QUEDA     AV EDVAL~
## # ... with 12,505 more rows, and 35 more variables: QUEDA_ARR <int>,
## #   DATA <date>, DATA_HORA <dttm>, DIA_SEM <chr>, HORA <time>,
## #   FERIDOS <int>, FERIDOS_GR <int>, MORTES <int>, MORTE_POST <int>,
## #   FATAIS <int>, AUTO <int>, TAXI <int>, LOTACAO <int>, ONIBUS_URB <int>,
## #   ONIBUS_MET <int>, ONIBUS_INT <int>, CAMINHAO <int>, MOTO <int>,
## #   CARROCA <int>, BICICLETA <int>, OUTRO <int>, TEMPO <chr>,
## #   NOITE_DIA <chr>, FONTE <chr>, BOLETIM <chr>, REGIAO <chr>, DIA <int>,
## #   MES <int>, ANO <int>, FX_HORA <int>, CONT_ACID <int>, CONT_VIT <int>,
## #   UPS <int>, CONSORCIO <chr>, CORREDOR <int>
\end{verbatim}

\paragraph{As we see, there is a lot of information here.Though at my
first look, I can't seem to find any relevant missing
data.}\label{as-we-see-there-is-a-lot-of-information-here.though-at-my-first-look-i-cant-seem-to-find-any-relevant-missing-data.}

Since for this first analysis we'll be trying to find out if there is a
time of the year with more accidents, we'll limit this dataset for this
pourpose.

\begin{Shaded}
\begin{Highlighting}[]
\NormalTok{ac_data }\OperatorTok\StringTok{  }
\StringTok{  }\KeywordTok{group_by}\NormalTok{(DATA) }\OperatorTok\StringTok{ }
\StringTok{  }\KeywordTok{summarise}\NormalTok{(}\DataTypeTok{QUANT_ACID =} \KeywordTok{sum}\NormalTok{(CONT_ACID)) }\OperatorTok
\StringTok{  }\KeywordTok{ggplot}\NormalTok{(}\KeywordTok{aes}\NormalTok{( }\DataTypeTok{x=}\NormalTok{DATA, }\DataTypeTok{y =}\NormalTok{QUANT_ACID))}\OperatorTok{+}\KeywordTok{geom_col}\NormalTok{() }\OperatorTok{+}
\StringTok{  }\KeywordTok{geom_point}\NormalTok{() }\OperatorTok{+}\StringTok{ }
\StringTok{  }\KeywordTok{ggtitle}\NormalTok{(}\StringTok{"Number of accidents by day / 2016"}\NormalTok{) }\OperatorTok{+}\StringTok{ }
\StringTok{  }\KeywordTok{xlab}\NormalTok{(}\StringTok{"Date"}\NormalTok{) }\OperatorTok{+}\StringTok{ }\KeywordTok{ylab}\NormalTok{(}\StringTok{"Number of Accidents"}\NormalTok{) }\OperatorTok{+}
\StringTok{  }\KeywordTok{theme_classic}\NormalTok{()}
\end{Highlighting}
\end{Shaded}

\includegraphics{Analysis_POA_Accidents_files/figure-latex/unnamed-chunk-3-1.pdf}

\paragraph{Although this graph shows a lot, it's better for us to
analyse and understand the relations between time of the year and
accidents if we look at the number of accidents per
month.}\label{although-this-graph-shows-a-lot-its-better-for-us-to-analyse-and-understand-the-relations-between-time-of-the-year-and-accidents-if-we-look-at-the-number-of-accidents-per-month.}

\begin{Shaded}
\begin{Highlighting}[]
\NormalTok{ac_data }\OperatorTok\StringTok{  }
\StringTok{  }\KeywordTok{group_by}\NormalTok{(MES) }\OperatorTok\StringTok{ }
\StringTok{  }\KeywordTok{summarise}\NormalTok{(}\DataTypeTok{QUANT_ACID =} \KeywordTok{sum}\NormalTok{(CONT_ACID)) }\OperatorTok
\StringTok{  }\KeywordTok{ggplot}\NormalTok{(}\KeywordTok{aes}\NormalTok{( }\DataTypeTok{x=}\NormalTok{MES, }\DataTypeTok{y =}\NormalTok{QUANT_ACID))}\OperatorTok{+}\KeywordTok{geom_col}\NormalTok{(}\DataTypeTok{count=}\DecValTok{12}\NormalTok{,}\DataTypeTok{binwidth =} \DecValTok{1}\NormalTok{) }\OperatorTok{+}
\StringTok{  }\KeywordTok{ggtitle}\NormalTok{(}\StringTok{"Number of accidents by month / 2016"}\NormalTok{) }\OperatorTok{+}\StringTok{ }
\StringTok{  }\KeywordTok{xlab}\NormalTok{(}\StringTok{"Month"}\NormalTok{) }\OperatorTok{+}\StringTok{ }\KeywordTok{ylab}\NormalTok{(}\StringTok{"Number of Accidents"}\NormalTok{) }\OperatorTok{+}
\StringTok{  }\KeywordTok{scale_x_discrete}\NormalTok{(}\DataTypeTok{limit =} \KeywordTok{c}\NormalTok{(}\StringTok{"Jan"}\NormalTok{, }\StringTok{"Feb"}\NormalTok{, }\StringTok{"Mar"}\NormalTok{, }\StringTok{"Apr"}\NormalTok{, }\StringTok{"May"}\NormalTok{, }\StringTok{"Jun"}\NormalTok{, }\StringTok{"Jul"}\NormalTok{, }\StringTok{"Aug"}\NormalTok{, }\StringTok{"Sep"}\NormalTok{, }\StringTok{"Oct"}\NormalTok{, }\StringTok{"Nov"}\NormalTok{, }\StringTok{"Dec"}\NormalTok{))}\OperatorTok{+}
\StringTok{  }\KeywordTok{theme_classic}\NormalTok{()}
\end{Highlighting}
\end{Shaded}

\begin{verbatim}
## Warning: Ignoring unknown parameters: count, binwidth
\end{verbatim}

\includegraphics{Analysis_POA_Accidents_files/figure-latex/unnamed-chunk-4-1.pdf}

\paragraph{Now, let's analyse to learn how many vehicles are usually
involved in the
accidents.}\label{now-lets-analyse-to-learn-how-many-vehicles-are-usually-involved-in-the-accidents.}

\begin{Shaded}
\begin{Highlighting}[]
\NormalTok{ac_data }\OperatorTok\StringTok{  }
\StringTok{  }\KeywordTok{group_by}\NormalTok{(AUTO) }\OperatorTok\StringTok{ }
\StringTok{  }\KeywordTok{summarise}\NormalTok{(}\DataTypeTok{QUANT_ACID =} \KeywordTok{sum}\NormalTok{(CONT_ACID)) }\OperatorTok
\StringTok{  }\KeywordTok{ggplot}\NormalTok{(}\KeywordTok{aes}\NormalTok{( }\DataTypeTok{x=}\NormalTok{AUTO, }\DataTypeTok{y =}\NormalTok{QUANT_ACID))}\OperatorTok{+}\KeywordTok{geom_col}\NormalTok{()}\OperatorTok{+}
\StringTok{  }\KeywordTok{theme_classic}\NormalTok{() }\OperatorTok{+}
\StringTok{  }\KeywordTok{ggtitle}\NormalTok{(}\StringTok{"Number of accidents by number of vehicles involved / 2016"}\NormalTok{) }\OperatorTok{+}\StringTok{ }
\StringTok{  }\KeywordTok{xlab}\NormalTok{(}\StringTok{"Number of Vehicles"}\NormalTok{) }\OperatorTok{+}\StringTok{ }\KeywordTok{ylab}\NormalTok{(}\StringTok{"Number of Accidents"}\NormalTok{) }\OperatorTok{+}
\StringTok{  }\KeywordTok{scale_x_continuous}\NormalTok{(}\DataTypeTok{breaks =} \KeywordTok{c}\NormalTok{(}\DecValTok{0}\NormalTok{,}\DecValTok{1}\NormalTok{,}\DecValTok{2}\NormalTok{,}\DecValTok{3}\NormalTok{, }\DecValTok{4}\NormalTok{), }\DataTypeTok{expand =} \KeywordTok{c}\NormalTok{(}\DecValTok{0}\NormalTok{,}\FloatTok{0.6}\NormalTok{)) }\OperatorTok{+}
\StringTok{  }\KeywordTok{scale_y_continuous}\NormalTok{(}\DataTypeTok{breaks =} \KeywordTok{c}\NormalTok{(}\DecValTok{0}\NormalTok{,}\DecValTok{1000}\NormalTok{,}\DecValTok{2000}\NormalTok{,}\DecValTok{3000}\NormalTok{,}\DecValTok{4000}\NormalTok{,}\DecValTok{5000}\NormalTok{, }\DecValTok{6000}\NormalTok{), }\DataTypeTok{expand =} \KeywordTok{c}\NormalTok{(}\DecValTok{0}\NormalTok{,}\DecValTok{0}\NormalTok{)) }\OperatorTok{+}
\StringTok{  }\KeywordTok{coord_cartesian}\NormalTok{(}\DataTypeTok{xlim =} \KeywordTok{c}\NormalTok{(}\DecValTok{0}\NormalTok{, }\DecValTok{4}\NormalTok{), }\DataTypeTok{ylim=}\KeywordTok{c}\NormalTok{(}\DecValTok{0}\NormalTok{,}\DecValTok{6000}\NormalTok{))}
\end{Highlighting}
\end{Shaded}

\includegraphics{Analysis_POA_Accidents_files/figure-latex/unnamed-chunk-5-1.pdf}

\paragraph{So we can see from here that most accidents happen envolving
1 or 2
vehicles.}\label{so-we-can-see-from-here-that-most-accidents-happen-envolving-1-or-2-vehicles.}

\paragraph{Now let's see if there is a certain weekday that has more
accidents than
others.}\label{now-lets-see-if-there-is-a-certain-weekday-that-has-more-accidents-than-others.}

\begin{Shaded}
\begin{Highlighting}[]
\NormalTok{positions <-}\StringTok{ }\KeywordTok{c}\NormalTok{(}\StringTok{"DOMINGO"}\NormalTok{, }\StringTok{"SEGUNDA-FEIRA"}\NormalTok{, }\StringTok{"TERCA-FEIRA"}\NormalTok{, }\StringTok{"QUARTA-FEIRA"}\NormalTok{, }\StringTok{"QUINTA-FEIRA"}\NormalTok{, }\StringTok{"SEXTA-FEIRA"}\NormalTok{, }\StringTok{"SABADO"}\NormalTok{) }
\NormalTok{ac_data }\OperatorTok\StringTok{ }
\StringTok{  }\KeywordTok{group_by}\NormalTok{(DIA_SEM) }\OperatorTok\StringTok{ }
\StringTok{  }\KeywordTok{summarise}\NormalTok{(}\DataTypeTok{QUANT_ACID =} \KeywordTok{sum}\NormalTok{(CONT_ACID)) }\OperatorTok
\StringTok{  }\KeywordTok{ggplot}\NormalTok{(}\KeywordTok{aes}\NormalTok{( }\DataTypeTok{x=}\NormalTok{DIA_SEM, }\DataTypeTok{y=}\NormalTok{QUANT_ACID))}\OperatorTok{+}\KeywordTok{geom_col}\NormalTok{()}\OperatorTok{+}\StringTok{ }
\StringTok{  }\KeywordTok{theme_classic}\NormalTok{() }\OperatorTok{+}
\StringTok{  }\KeywordTok{ggtitle}\NormalTok{(}\StringTok{"Number of accidents by day of the week / 2016"}\NormalTok{) }\OperatorTok{+}\StringTok{ }
\StringTok{  }\KeywordTok{xlab}\NormalTok{(}\StringTok{"Day of the week"}\NormalTok{) }\OperatorTok{+}\StringTok{ }\KeywordTok{ylab}\NormalTok{(}\StringTok{"Number of Accidents"}\NormalTok{) }\OperatorTok{+}
\StringTok{  }\KeywordTok{scale_x_discrete}\NormalTok{(}\DataTypeTok{limits=}\NormalTok{ positions, }
                   \DataTypeTok{labels=}\KeywordTok{c}\NormalTok{(}\StringTok{"DOMINGO"}\NormalTok{=}\StringTok{"SUNDAY"}\NormalTok{,}\StringTok{"SEGUNDA-FEIRA"}\NormalTok{=}\StringTok{"MONDAY"}\NormalTok{, }\StringTok{"TERCA-FEIRA"}\NormalTok{=}\StringTok{"TUESDAY"}\NormalTok{, }
                            \StringTok{"QUARTA-FEIRA"}\NormalTok{ =}\StringTok{ "WEDNESDAY"}\NormalTok{, }\StringTok{"QUINTA-FEIRA"}\NormalTok{ =}\StringTok{ "THURSDAY"}\NormalTok{, }
                            \StringTok{"SEXTA-FEIRA"}\NormalTok{ =}\StringTok{ "FRIDAY"}\NormalTok{, }\StringTok{"SABADO"}\NormalTok{ =}\StringTok{ "SATURDAY"}\NormalTok{))}
\end{Highlighting}
\end{Shaded}

\includegraphics{Analysis_POA_Accidents_files/figure-latex/unnamed-chunk-6-1.pdf}

\paragraph{From this graph we can certainly observe some interesting
things. The first thing that comes to mind is that there are more
accidents on Fridays, usually when people go out to party. And the
number of accidents on Saturdays and Sundays are low, maybe because
people tend to stay at home during those
days.}\label{from-this-graph-we-can-certainly-observe-some-interesting-things.-the-first-thing-that-comes-to-mind-is-that-there-are-more-accidents-on-fridays-usually-when-people-go-out-to-party.-and-the-number-of-accidents-on-saturdays-and-sundays-are-low-maybe-because-people-tend-to-stay-at-home-during-those-days.}

\paragraph{Another interesting thing to look at is in what days the
percentage of fatal accidents is
higher.}\label{another-interesting-thing-to-look-at-is-in-what-days-the-percentage-of-fatal-accidents-is-higher.}

\begin{Shaded}
\begin{Highlighting}[]
\NormalTok{positions <-}\StringTok{ }\KeywordTok{c}\NormalTok{(}\StringTok{"DOMINGO"}\NormalTok{, }\StringTok{"SEGUNDA-FEIRA"}\NormalTok{, }\StringTok{"TERCA-FEIRA"}\NormalTok{, }\StringTok{"QUARTA-FEIRA"}\NormalTok{, }\StringTok{"QUINTA-FEIRA"}\NormalTok{, }\StringTok{"SEXTA-FEIRA"}\NormalTok{, }\StringTok{"SABADO"}\NormalTok{) }
\NormalTok{ac_data }\OperatorTok\StringTok{ }
\StringTok{  }\KeywordTok{group_by}\NormalTok{(DIA_SEM) }\OperatorTok\StringTok{ }
\StringTok{  }\KeywordTok{summarise}\NormalTok{(}\DataTypeTok{QUANT_ACID =} \KeywordTok{sum}\NormalTok{(CONT_ACID), }\DataTypeTok{Prcnt_fatal =} \KeywordTok{sum}\NormalTok{(FATAIS)}\OperatorTok{/}\KeywordTok{sum}\NormalTok{(CONT_ACID)}\OperatorTok{*}\DecValTok{100}\NormalTok{) }\OperatorTok
\StringTok{  }\KeywordTok{ggplot}\NormalTok{(}\KeywordTok{aes}\NormalTok{( }\DataTypeTok{x=}\NormalTok{DIA_SEM, }\DataTypeTok{y=}\NormalTok{QUANT_ACID, }\DataTypeTok{fill=}\NormalTok{Prcnt_fatal))}\OperatorTok{+}\KeywordTok{geom_col}\NormalTok{()}\OperatorTok{+}\StringTok{ }
\StringTok{  }\KeywordTok{scale_fill_gradient}\NormalTok{(}\DataTypeTok{low=}\StringTok{"yellow"}\NormalTok{, }\DataTypeTok{high=}\StringTok{"red"}\NormalTok{) }\OperatorTok{+}
\StringTok{  }\KeywordTok{theme_classic}\NormalTok{() }\OperatorTok{+}
\StringTok{  }\KeywordTok{ggtitle}\NormalTok{(}\StringTok{"Number of accidents by day of the week / 2016"}\NormalTok{) }\OperatorTok{+}\StringTok{ }
\StringTok{  }\KeywordTok{xlab}\NormalTok{(}\StringTok{"Day of the week"}\NormalTok{) }\OperatorTok{+}\StringTok{ }\KeywordTok{ylab}\NormalTok{(}\StringTok{"Number of Accidents"}\NormalTok{) }\OperatorTok{+}
\StringTok{  }\KeywordTok{scale_x_discrete}\NormalTok{(}\DataTypeTok{limits=}\NormalTok{ positions, }
                   \DataTypeTok{labels=}\KeywordTok{c}\NormalTok{(}\StringTok{"DOMINGO"}\NormalTok{=}\StringTok{"SUNDAY"}\NormalTok{,}\StringTok{"SEGUNDA-FEIRA"}\NormalTok{=}\StringTok{"MONDAY"}\NormalTok{, }\StringTok{"TERCA-FEIRA"}\NormalTok{=}\StringTok{"TUESDAY"}\NormalTok{, }
                            \StringTok{"QUARTA-FEIRA"}\NormalTok{ =}\StringTok{ "WEDNESDAY"}\NormalTok{, }\StringTok{"QUINTA-FEIRA"}\NormalTok{ =}\StringTok{ "THURSDAY"}\NormalTok{, }
                            \StringTok{"SEXTA-FEIRA"}\NormalTok{ =}\StringTok{ "FRIDAY"}\NormalTok{, }\StringTok{"SABADO"}\NormalTok{ =}\StringTok{ "SATURDAY"}\NormalTok{)) }\OperatorTok{+}\StringTok{ }
\StringTok{  }\KeywordTok{guides}\NormalTok{(}\DataTypeTok{fill =} \KeywordTok{guide_legend}\NormalTok{(}\DataTypeTok{title =} \StringTok{"Fatal Percentage"}\NormalTok{, }\DataTypeTok{label.position =} \StringTok{"left"}\NormalTok{, }\DataTypeTok{title.theme=}\KeywordTok{element_text}\NormalTok{(}\DataTypeTok{size=}\DecValTok{9}\NormalTok{)))}
\end{Highlighting}
\end{Shaded}

\includegraphics{Analysis_POA_Accidents_files/figure-latex/unnamed-chunk-7-1.pdf}
\#\#\#\# We conclude by the graph above that even if the number of
accidents is lower, the percentage of fatal accidents on weekends is far
higher than the percentage in weekdays.

\paragraph{\texorpdfstring{So, lets see if there are regions in Porto
Alegre with more accidents than others. For this, I define ``Region'' as
the column
`REGIAO'.}{So, lets see if there are regions in Porto Alegre with more accidents than others. For this, I define Region as the column REGIAO.}}\label{so-lets-see-if-there-are-regions-in-porto-alegre-with-more-accidents-than-others.-for-this-i-define-region-as-the-column-regiao.}

\begin{Shaded}
\begin{Highlighting}[]
\NormalTok{ac_data }\OperatorTok\StringTok{  }
\StringTok{  }\KeywordTok{subset}\NormalTok{(REGIAO }\OperatorTok{!=}\StringTok{ "NAO IDENTIFICADO"}\NormalTok{) }\OperatorTok
\StringTok{  }\KeywordTok{group_by}\NormalTok{(REGIAO) }\OperatorTok\StringTok{ }
\StringTok{  }\KeywordTok{summarise}\NormalTok{(}\DataTypeTok{QUANT_ACID =} \KeywordTok{sum}\NormalTok{(CONT_ACID)) }\OperatorTok
\StringTok{  }\KeywordTok{ggplot}\NormalTok{(}\KeywordTok{aes}\NormalTok{( }\DataTypeTok{x=}\NormalTok{REGIAO, }\DataTypeTok{y=}\NormalTok{QUANT_ACID))}\OperatorTok{+}\KeywordTok{geom_col}\NormalTok{()}\OperatorTok{+}
\StringTok{  }\KeywordTok{theme_classic}\NormalTok{() }\OperatorTok{+}
\StringTok{  }\KeywordTok{ggtitle}\NormalTok{(}\StringTok{"Number of accidents by region / 2016"}\NormalTok{) }\OperatorTok{+}\StringTok{ }
\StringTok{  }\KeywordTok{xlab}\NormalTok{(}\StringTok{"Region"}\NormalTok{) }\OperatorTok{+}\StringTok{ }\KeywordTok{ylab}\NormalTok{(}\StringTok{"Number of Accidents"}\NormalTok{)}
\end{Highlighting}
\end{Shaded}

\includegraphics{Analysis_POA_Accidents_files/figure-latex/unnamed-chunk-8-1.pdf}


\end{document}
